\documentclass[twoside,11pt]{article}
\usepackage[top=1in,bottom=1in,left=1in,right=1in]{geometry}
\usepackage{graphicx,bm,amssymb,breqn}
\usepackage[sort&compress]{natbib} \setlength{\bibsep}{0.0pt}
\usepackage{algorithm,algpseudocode}

% colors
\usepackage{color}
\definecolor{darkred}{RGB}{100,0,0}
\definecolor{darkgreen}{RGB}{0,100,0}
\definecolor{darkblue}{RGB}{0,0,150}

% links
\usepackage{hyperref}
\hypersetup{colorlinks=true, linkcolor=darkred, citecolor=darkgreen, urlcolor=darkblue}
\usepackage{url}

\pagestyle{myheadings}

\renewcommand{\algorithmicrequire}{\textbf{Input:}}
\renewcommand{\algorithmicensure}{\textbf{Output:}}

\newcommand{\pp}{\mathbb} %Spaces notation
\newcommand{\cc}{\mathcal} %Set notation

\def\beq{\begin{equation}} % \setcounter{equation}{1}}
\def\eeq{\end{equation}}
\def\beqn{\begin{dmath*}}
\def\eeqn{\end{dmath*}}
\newcommand{\teq}[1]{\smash{$#1$}}
\newcommand{\logit}{{\rm logit}}


\begin{document}


\title{Learning Attachment Preference Trends in Dynamic Networks}
\author{
Andrew Chin$^{1}$\\
{\tt aschin@ucdavis.edu}\\
\and
James Sharpnack$^{1}$ \\
{\tt  jsharpna@ucdavis.edu} \\
\and
\begin{tabular}{c}
  $^{1}$ Department of Statistics, UC Davis, Davis, CA 95616\\
\end{tabular}
}

\date{}%\today}
\maketitle

\begin{abstract}
Here is the abstract.

\medskip\noindent
{\em Keywords:} keyword 1, keyword 2.
\end{abstract}


\section{Introduction}
\label{sec:intro}

\subsection{Notation}
Vectors and matrices are boldfaced $\bm a, \bm A$.
Sets and spaces are caligraphic and uppercase $\cc S$, probability measures and some sets are bold-faced, $\pp P$.

\subsection{Trend-based Model}

At each time point \teq t, we observe a graph \teq{G_t}, such at time \teq t a new vertex is generated.
It will be assumed that only one vertex is generated at a time, meaning that two vertices do not appear simultaneously.
The precise time at which the new vertex is generated will be ignored, and the only the changes in network topology  between vertex GENERATION EVENTS is considered informative.
This can be thought of as the jump process for the dynamic network, and henceforth we will assume that our dynamic network is in discrete time, with a new vertex added at each time increment.
Thus, we let $v_t$ denote the vertex added at time $t$, and $\cc V_{t} = \{v_1,\ldots,v_t \}$ denotes the set of vertices up to time $t$ (let $n_t$ be the number of vertices of $G_t$).

Furthermore, the only changes to the edge set at time $t$ are assumed to be the edges incident to $v_t$, and otherwise the edges persist throughout time.
The goal of this work is to uncover active regions in the network.
LEAD IN TO ATTACHMENT
The event in which vertex $w \in \cc V_{t-1}$ attaches to $v_t$ is denoted $A_t(w)$, and we assume that $A_t(w)$ are independent Bernoulli trials for all $w$.
We further assume that the model is Markovian, in that $A_t(w)$ is independent of $G_1,\ldots, G_{t-2}$ conditional on $G_{t-1}$.
Hence, the joint distribution of $A_t(w)|G_{t-1}$ is uniquely identified with a probability vector $\bm p_t \in \pp R^{n_t - 1}$, where $\pp P(A_t(w) = 1) = p_{t,w}$.

It is common to describe probabilities for Bernoulli events through a link function, $g(p)$ which is often taken to be the logistic function \teq{\logit(p) = \log\left(\frac{p}{1-p}\right)}.
So, let us decompose
\beq
\logit(\bm p_t) = \bm a_t + a_0.
\eeq
where logit is applied elementwise, $a_0$ is a scalar, and $\sum_w \bm a_{t,w} = 0$.
In words, we will call the vector $\bm a_t$ the {\em trend vector} at time $t$.
Roughly, vertices, $w$, with large values of $a_{t,w}$ are thought to be `trending' in the sense that new vertices will link to or be linked to by $w$.
The intercept term $a_0$ is thought of as a baseline probability of attachment, and in sparse networks it will be negative.




We will assume that the dynamic network belongs to a Markov model.
For each time point, $t$, we use $G_t$, to denote the network at that time.

If we condition on the current realization 

\beq 
\label{eq:eqname}
\bm \alpha \star \bm \beta
\eeq
and inline equations are \teq{\sum \alpha + \beta}


\subsection*{Acknowledgments}

Grant support, people that helped.

\bibliographystyle{chicago}
\bibliography{biblio}

\end{document}
